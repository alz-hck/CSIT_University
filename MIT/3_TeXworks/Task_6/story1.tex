Теперь я расскажу о том, как я родился, как я рос и как обнаружились во мне первые признаки гения. Я родился дважды. Произошло это вот как.

Мой папа женился на моей маме в 1902 году, но меня мои родители произвели на свет только в конце 1905 года, потому что папа пожелал, чтобы его ребенок родился обязательно на Новый год. Папа рассчитал, что зачатие должно произойти 1-го апреля и только в этот день подъехал к маме с предложением зачать ребенка.

Первый раз папа подъехал к моей маме 1-го апреля 1903-го года. Мама давно ждала этого момента и страшно обрадовалась. Но папа, как видно, был в очень шутливом настроении и не удержался и сказал маме: <<С первым апреля!>>

Мама страшно обиделась и в этот день не подпустила папу к себе. Пришлось ждать до следующего года.

В 1904 году, 1-го апреля, папа начал опять подъезжать к маме с тем же предложением. Но мама, помня прошлогодний случай, сказала, что теперь она уже больше не желает оставаться в глупом положении, и опять не подпустила к себе папу. Сколько папа ни бушевал, ничего не помогло.

И только год спустя удалось моему папе уломать мою маму и зачать меня.

Итак мое зачатие произошло 1-го апреля 1905 года.

Однако все папины расчеты рухнули, потому что я оказался недоноском и родился на четыре месяца раньше срока.

Папа так разбушевался, что акушерка, принявшая меня, растерялась и начала запихивать меня обратно, откуда я только что вылез.

Присутствующий при этом один наш знакомый, студент Военно-Медицинской Академии, заявил, что запихать меня обратно не удастся. Однако несмотря на слова студента, меня все же запихали, но, правда, как потом выяснилось, запихать-то запихали, да второпях не туда.

Тут началась страшная суматоха.

Родительница кричит: <<Подавайте мне моего ребенка!>> А ей отвечают: <<Ваш, говорят, ребенок находится внутри вас>>. <<Как!>>\ ---~кричит родительница. <<Как ребенок внутри меня, когда я его только что родила!>>

<<Но>>, \ ---~говорят родительнице, <<может быть вы ошибаетесь?>> <<Как!>> \ ---~кричит родительница, <<ошибаюсь! Разве я могу ошибаться! Я сама видела, что ребенок только что вот тут лежал на простыне!>> <<Это верно>>, \ ---~говорят родительнице. <<Но, может быть, он куда-нибудь заполз>>. Одним словом, и сами не знают, что сказать родительнице.

А родительница шумит и требует своего ребенка.

Пришлось звать опытного доктора. Опытный доктор осмотрел родительницу и руками развел, однако все же сообразил и дал родительнице хорошую порцию английской соли. Родительницу пронесло, и таким образом я вторично вышел на свет.

Тут опять папа разбушевался\ ---~дескать, это, мол, еще нельзя назвать рождением, что это, мол, еще не человек, а скорее наполовину зародыш, и что его следует либо опять обратно запихать, либо посадить в инкубатор.

И они посадили меня в инкубатор.

25 сентября 1935 года.