\textbf{Водевиль в четырех частях}

\textbf{Цена 30 рублей}

\textbf{Часть первая}

АНТОН ИСААКОВИЧ. Не хочу больше быть Антоном, а хочу быть Адамом. А ты, Наташа, будь Евой.

НАТАЛИЯ БОРИСОВНА (сидя на кордонке с халвой). Да ты что: с ума сошел?

АНТОН ИСААКОВИЧ. Ничего я с ума не сошел. Я буду Адамом, а ты будешь Ева!

НАТАЛИЯ БОРИСОВНА (смотря налево и направо). Ничего не понимаю!

АНТОН ИСААКОВИЧ. Это очень просто! Мы встанем на письменный стол, и, когда кто-нибудь будет входить к нам, мы будем кланяться и говорить: <<Разрешите представиться\ ---~Адам и Ева>>.

НАТАЛИЯ БОРИСОВНА. Ты сошел с ума! Ты сошел с ума!

АНТОН ИСААКОВИЧ (влезая на письменный стол и таща за руку Наталию Борисовну). Ну вот будем тут стоять и кланяться пришедшим.

НАТАЛИЯ БОРИСОВНА (залезая на письменный стол). Почему? Почему?

АНТОН ИСААКОВИЧ. Ну вот, слышишь два звонка! Это к нам приготовься.

В дверь стучат.

Войдите!

Входит Вейсбрем.

АНТОН ИСААКОВИЧ и НАТАЛИЯ БОРИСОВНА (кланяясь). Разрешите представиться: Адам и Ева!

Вейсбрем падает как пораженный громом.

\textbf{Занавес}

\textbf{Часть вторая}

По улице скачут люди на трех ногах. Из Москвы дует фиолетовый ветер.

\textbf{Занавес}

\textbf{Часть третья}

Адам Исаакович и Ева Борисовна летают над городом Ленинградом. Народ стоит на коленях и просит о пощаде. Адам Исаакович и Ева Борисовна добродушно смеются.

\textbf{Занавес}

\textbf{Часть четвертая и последняя}

Адам и Ева сидят на березе и поют.

\textbf{Занавес}

23 февраля 1935 года.